В результате данной работы была предложена оценка максимального правдоподобия $\theta^\ast_n$ параметра хвоста $\overline{F}_\theta$ распределения $F_\theta$ по выборке $X_1,\:\ldots,\: X_n$ из этого распределения.

В ходе работы были проанализированы существующие результаты теории экстремумов. 

В разделе 3 было доказано экстремальное свойство правдоподобия $k$ верхних порядковых статистик $X_{(n-k+1)},\:\ldots,\: X_{(n)}$ выборки $X_1,\:\ldots,\: X_n$.

В разделе 4 были найдены условия, при которых полученная оценка $\theta^\ast_n$  является состоятельной оценкой параметра хвоста распределения. 

В разделе 5 была доказана асимптотическая нормальность оценки $\theta^\ast_n$ (при дополнительных предположениях).

В разделе 6 были сформулированы условия гарантирующие выполнение дополнительных предположений предыдущего раздела, а также найдена асимптотика выражений, возникающих при доказательстве сходимости предложенной оценки к нормальному закону. 




 
\documentclass[../manuscript.tex]{subfiles}

\section{Благодарности}
Данная работа посвящена приложениям байесовских методов машинного обучения к актуальной задаче вычислительной онкологии — задаче восстановления клональной структуры опухоли по данным высокопроизводительного секвенирования одиночных клеток. Такой выбор темы отражает научные интересы автора — статистическое моделирование на больших медицинских данных, — сформировавшиеся за время работы над проектами по вычислительной системной биологии под руководством к.ф.-м.н. Юрия Львовича Притыкина\footnote{https://scholar.google.com/citations?user=Arx56RkJBrYC\&hl=en}. Автор благодарен ему за время, уделённое на протяжении двух лет работы под его руководством, и за возможность уже на младших курсах приобщиться к высоким стандартам современной академической науки.

Работа над дипломом велась под руководством Оливера Штегле\footnote{https://scholar.google.com/citations?user=ClSXZ4IAAAAJ\&hl=en}, профессора Heidelberg University\footnote{https://www.uni-heidelberg.de/en} (Университет Хайдельберга, Германия) а также действующих и бывших сотрудников его научных групп в DKFZ\footnote{https://www.dkfz.de/en/index.html} (Немецкий Центр Онкологических Исследований), EMBL Heidelberg\footnote{https://www.embl.de/} (Европейская Лаборатория Молекулярной Биологии) и HKU\footnote{https://www.hku.hk/} (Университет Гонконга). Формальным научным руководителем следует считать к.ф.-м.н. Yuanhua Huang\footnote{https://www.sbms.hku.hk/staff/yuanhua-huang}, заведующего группой вычислительной биологии в Университете Гонконга. Автор признателен ему за профессионализм и тщательность, с которой он на протяжении многих месяцев руководил разработкой метода. Кроме того, автор выражает личную благодарность профессору Штегле и к.ф.-м.н. Ханне Сьюзак, Николе Казирахи, Родриго Гонцало Парра, а также Д. О. Бредихину и В. А. Огородникову за ценные замечания и советы, придавшие многим аспектам метода завершённый, логически стройный вид.

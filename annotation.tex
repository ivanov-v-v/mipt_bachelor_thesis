\documentclass[../main.tex]{subfiles}

\section{Аннотация}

В данной дипломной работе предложена графическая байесовская модель машинного обучения XClone. Её задача — восстановление клонального состава опухоли по данным ДНК-секвенирования одиночных клеток. XClone — статистическая модель опухолевого образца, оптимальные параметры которой подбираются посредством вариационного байесовского вывода. По этим параметрам можно восстановить структурные мутации на каждой из хромосом в клеток образца, что позволяет проследить эволюцию опухоли. В работе описана формальная постановка задачи и приведена реализация алгоритма на языке программирования Python. Актуальность задачи подтверждается тем, что в последние годы несколько статей схожей тематики — SiCloneFit\cite{SiCloneFit}, InferCNV\footnote{https://github.com/broadinstitute/inferCNV/wiki}, Casper\cite{Casper}, CHISEL\cite{ChiselBiorxiv} — было опубликовано в высокоимпактных научных журналах, но среди них не было полных аналогов. Практическую ценность работы подтверждает то, что задача пришла из клинической практики немецких врачей-онкологов. Научная новизна работы заключается в том, что, несмотря на популярность темы, у него пока есть всего один прямой конкурент — алгоритм CHISEL, — от которого XClone выгодно отличается тем, что допускает естественное обобщение на случай нескольких модальностей. Концептуально XClone может поддерживать не только геномные, но и транскриптомные данные, а также информацию о соматических мутациях и митохондриальной ДНК в клетках образца. Каждая из этих модальностей имеет клиническую ценность и позволяет лучше понимать эволюцию опухоли.
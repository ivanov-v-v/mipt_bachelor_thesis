\documentclass[../main.tex]{subfiles}

\section{Аннотация}

В данной дипломной работе предложена графическая байесовская модель машинного обучения XClone. Её задача — восстановление клонального состава опухоли по данным ДНК-секвенирования одиночных клеток. XClone — статистическая модель опухолевого образца, оптимальные параметры которой подбираются посредством вариационного байесовского вывода. По этим параметрам можно восстановить структурные мутации на каждой из хромосом в каждой из клеток образца, что позволяет проследить эволюцию опухоли. 

В работе описана формальная постановка задачи и приведена реализация алгоритма на языке программирования Python. Актуальность задачи подтверждается тем, что в последние годы несколько статей схожей тематики — SiCloneFit\cite{SiCloneFit}, InferCNV\footnote{https://github.com/broadinstitute/inferCNV/wiki}, CaSpER\cite{Casper}, CHISEL\cite{ChiselBiorxiv} — было опубликовано в высокоимпактных научных журналах, но среди них не было полных аналогов. Практическую ценность работы подтверждает то, что задача пришла из клинической практики немецких врачей-онкологов. Научная новизна работы заключается в том, что, несмотря на популярность темы, у XClone пока есть всего один прямой конкурент — алгоритм CHISEL, — от которого XClone выгодно отличается как производительностью, так и тем, что допускает естественное обобщение на несколько модальностей, каждая из которых уточняет диагноз: scRNA-seq, scATAC-seq, митохондриальные геномы, соматические мутации. 

Ведётся активная работа над тем, чтобы сделать XClone одним из первых алгоритмов поиска аллель-зависимых структурных вариаций в данных РНК-секвенирования одиночных клеток. С этой целью намечена коллаборация с учёными из университета Гонконга и из Стэнфорда с апробацией алгоритма XClone на их экспериментальных данных.
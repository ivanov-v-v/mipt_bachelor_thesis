\documentclass[../manuscript.tex]{subfiles}

\section{Материалы и методы}
\subsection{Использованные данные}
\subsection{Алгоритмы предобработки данных}
\subsubsection{Извлечение данных из BAM-файлов}
\subsubsection{Статистическое гаплотипирование ОНП}
\subsubsection{Подходы к сегментации генома}
\subsubsection{Pileup — подсчёт ридов, выравнивающихся на сегменты}
\subsubsection{Исправление ошибок переключения}
Поскольку одной из главных задач XClone является предсказание \textit{аллель-специфичных} структурных вариаций в геноме, матрицы $\mathrm{AD}$ и $\mathrm{DP}$ аллель-специфичных прочтений должны отражать биологию аллельного дисбаланса в клетках образца. Для этого нужно понимать, к какому гаплотипу принадлежит каждый ОНП. В разделе про статистическое гаплотипирование ОНП был сделан акцент на том, что существующие алгоритмы гарантируют только локальную корректность: при использовании алгоритма EAGLE2, следует ожидать, что при разбиении хромосомы на непересекающиеся окна длины 20-50 килобаз все гетерозиготные ОНП в пределах одного окна будут иметь одинаковый гаплотип, если это на самом деле так. Тем не менее, гаплотипы соседних сегментов с точки зрения алгоритма могут не совпадать даже тогда, когда на самом деле должны. К этому приводят так называемые \textbf{ошибки переключения} — спонтанная и неявная замена гаплотипических меток на противоположные внутри алгоритма. Классификацию ошибок переключения можно найти в статье \cite{Choi2018Plos}, цитата из которой приведена ниже:

\textit{"Phasing accuracy is typically measured by counting the number of \textbf{‘switches’} between known maternal and paternal haplotypes that should not occur if individual maternal and paternal chromosomal nucleotide sequence content has been accurately characterized. If an inconsistency is identified, then it is called a ‘switch error.’ These switch errors manifest themselves as induced and false recombination events in the inferred haplotypes compared with the true haplotypes. To identify \textbf{switch errors}, the phase of each site is compared with upstream neighboring phased sites. The switch error rate (SER) is defined as the number of switch errors divided by the number of opportunities for switch errors. Switch errors were further classified into three categories: \textbf{long}, \textbf{point}, and \textbf{undetermined}. A long switch appears as a large-scale pseudo recombination event; that is, there are no other switches in the local neighborhood around the long switch (e.g., no other switches within three consecutive heterozygous sites). On the contrary, a small-scale switch error appearing as two neighboring switch errors is considered as a point switch (e.g., two switches within three consecutive heterozygous sites, with the pair of switches counted as a point switch). The remaining switches are considered undetermined (e.g., only two sites phased in a small phasing block, so the switch error could not be classified into long or point)."}

Тем не менее, разбиение генома на фрагменты по 20-50 килобаз непрактично: в силу разреженности данных, в каждом таком сегменте может оказаться всего несколько ридов. Это даёт очень слабый и шумный сигнал аллельного дисбаланса. В связи с этим был разработан метод, одновременно решающий обе описанные проблем. На первом шаге алгоритма происходит разбиение генома на непересекающиеся сплошные сегменты длины $L$. Затем каждые $N$ подряд идущих сегментов объединяются в блок длины $NL$. В пределах блока переключения моделируются бернуллиевскими случайными величинами, по одной на каждый сегмент. Параметры этих распределений, в свою очередь, выводятся \textbf{ЕМ-алгоритмом}. После исправления ошибок, прочтения сегментов внутри блока суммируются, что даёт более стабильный сигнал. Эта идея была сформулирована в \cite{ChiselBiorxiv}, но технические детали были осознанно исключены авторами CHISEL из препринта. 

Прежде чем приступать к рассмотрению метода, сформулируем необходимые определения:

\begin{definition}[\textit{ЕМ-алгоритм}]$  $\\
EM-алгоритм (от английского \textit{"EM"} — \textit{"Expectation Maximization"}) — метод поиска оценок максимального правдоподобия (ОМП) или оценок апостериорного максимума (ОАП) параметров статистических моделей, содержащих скрытые переменные. 

\begin{algorithm}[H]
	\SetAlgoLined
	\KwResult{$\bm{\Theta}^{\ast},\ p(\bm{Z}\ |\ \bm{X}, \bm{\Theta}^{\ast})$}
	\text{$t = 0$}\;
	\text{$\bm{\Theta}^{(0)}$ инициализируется случайно}\;
	\While{$Q(\bm{\Theta}^{(t+1)}\ |\ \bm{\Theta}^{(t+1)}) - Q(\bm{\Theta}^{(t)}\ |\ \bm{\Theta}^{(t)}) > \eps$}{
		$\mathcal{L}(\bm{\Theta}^{(t)}; \bm{Z}, \bm{X}) := p(\bm{X}, \bm{Z}\ |\ \bm{\Theta}^{(t)})$\;
		$Q(\bm{\Theta}\ |\ \bm{\Theta}^{(t)}) := \mathbb{E}_{\bm{Z} | \bm{X}, \bm{\Theta}^{(t)}} \log \mathcal{L}(\bm{\Theta}; \bm{Z}, \bm{X})$ \tcp{E-шаг}
		$\bm{\Theta}^{(t+1)} := \arg\max\limits_{\bm{\Theta}} Q(\bm{\Theta}\ |\ \bm{\Theta}^{(t)})$ \tcp{M-шаг}
		$t = t + 1$
	}
	$\bm{\Theta}^{\ast} := \bm{\Theta}^{(t)}$
	\caption{ЕМ-алгоритм в общем виде}
\end{algorithm}

Здесь $\bm{Z}$ — дискретные скрытые переменные, $\bm{\Theta}$ — параметры статистической модели, $\bm{X}$ — выборка, $\eps > 0$. Каждая итерация алгоритма состоит из двух основных шагов:
\begin{enumerate}
	\item \textbf{Е-шаг}, на котором устраняется явная зависимость от скрытых переменных посредством взятия матожидания логарифма совместной функции правдоподобия по условному распределению $\bm{Z}\ |\ \bm{X}, \bm{\Theta}^{(t)}$;
	\item \textbf{M-шаг}, на котором параметры нового апостериорного распределения $\bm{\Theta}^{(t+1)}$ выбираются таким образом, чтобы максимизировать $ Q(\bm{\Theta}, \bm{\Theta}^{(t)}) $ — функцию правдоподобия "в среднем". 
\end{enumerate}
\end{definition}

С теоретическим обоснованием и формальным доказательством корректности EM-алгоритма можно ознакомиться в (\cite{MurphyProbabilisticML}, стр. 363-365). В контексте решаемой задачи $\bm{X}, \bm{Z}, \bm{\Theta}$ имеют следующий смысл:
\begin{itemize}
	\item $\bm{Z} = \{z_{1}, \ldots, z_{N}\}$ — независимые в совокупности индикаторы корректности гаплотипов сегментов
	\begin{gather*}
			\forall i: z_i \sim \mathrm{Bern}(p_i)\\
			\forall q \in \{0, 1\}^{N}: p(\bm{Z} = q\ |\ p_1, \ldots, p_n) = \prod_{i=1}^{N} p(z_i = q_i\ |\ p_i) = \prod_{i=1}^{N} p_i^{q_i} ( 1 - p_i)^{1-q_i}
	\end{gather*}
	Если $z_i = 1$, то будем говорить, что сегмент $i$ имеет корректный гаплотип, иначе — инвертированный. Эти обозначения имеют смысл только в пределах одного блока, в соседних блоках метки могут иметь противоположный смысл. Из этого наблюдения становится ясно, что алгоритм не решает проблему переключения полностью, но уменьшает число ошибок за счёт аггрегации сегментов в блоки.
	\item Обозначим через $M$ число клеток образца, тогда $\bm{X} = (\bm{X}_{1}, \ldots, \bm{X}_{M}),\ X_{c} := (\bm{a}_c, \bm{b}_c)$ —  вектора прочтений для каждой из клеток, по компоненте на сегмент.  $\bm{a}_c = (a_{c, 1}, \ldots, a_{c, N})$ — число прочтений аллеля А, $\bm{b}_c =  (b_{c, 1}, \ldots, b_{c, N})$ — аллеля Б.
	\item $\forall c \in \overline{1, M}: \bm{r}_c := \bm{a}_{c} + \bm{b}_{c}$ — вектора прочтений обоих аллелей вместе.
	\item $\bm{\Theta} = (\theta_1, \ldots, \theta_M; p_1, \ldots, p_N)$, где $\theta_c$ — пропорция ридов гаплотипа 1 в блоке в клетке $c$. Алгоритм предполагает, что пропорция гаплотипа 1 одинакова во всех сегментах внутри блока с точностью до переключения.
\end{itemize}
	
В этих обозначениях можно сформулировать и доказать следующее утверждение:
\begin{claim}
	Правила пересчёта параметров апостериорного распределения на М-шаге ЕМ-алгоритма имеют вид:
	\begin{equation}
		\begin{aligned}
			p_{i}^{(t+1)} &= \dfrac{ p_{i}^{(t)} \prod_{c=1}^{M} (\theta_{c}^{(t)})^{a_{c, i}} (1 - \theta_{c}^{(t)})^{b_{c, i}}}{p_{i}^{(t)} \prod_{c=1}^{M} (\theta_{c}^{(t)})^{a_{c, i}} (1 - \theta_{c}^{(t)})^{b_{c, i}} + (1 - p_{i}^{(t)}) \prod_{c=1}^{M} (\theta_{c}^{(t)})^{b_{c, i}} (1 - \theta_{c}^{(t)})^{a_{c, i}}}\\
			\theta_c^{(t+1)} &= \frac{\sum_{i=1}^{N}a_{i,c} \gamma^{(t)}_{i,1} + b_{i,c} \gamma^{(t)}_{i,0}}{\sum_{i=1}^{N} r_{i,c}}
		\end{aligned}
		\label{em_update_rules}
	\end{equation}
	где $\forall j \in \{0, 1\}: \gamma^{(t)}_{i,j} := P(z_i = j\ |\ \bm{X}, \bm{\Theta}^{(t)})$.
\end{claim}
\begin{proof}
  Вектора прочтений в клетках независимы в совокупности, потому: 
	\begin{gather*}
			P(\bm{X}\ |\ \bm{Z}, \bm{\Theta}) = \prod_{c=1}^{M} p(\bm{X}_c\ |\ \bm{Z}, \bm{\Theta}) = \prod_{c=1}^{N} \theta_{c}^{\widehat{a}_c(\bm{Z})}(1 - \theta_{c})^{\widehat{b}_c(\bm{Z})}
	\end{gather*}
	Где 
	\begin{gather*}
	\begin{cases}
		\widehat{a}_c(\bm{Z}) := \sum_{i=1}^{N} \left[z_i a_{c, i} + (1 - z_i) b_{c, i} \right],\\
		\widehat{b}_c(\bm{Z}) := \sum_{i=1}^{N} \left[(1 - z_i) a_{c, i} + z_i b_{c, i} \right],\\
		c \in \overline{1, M}
	\end{cases}
	\end{gather*}
	Тогда функция правдоподобия и её логарифм принимают вид
	\begin{align*}
			&\mathcal{L}(\bm{\Theta}; \bm{X}, \bm{Z}) = p(\bm{X}, \bm{Z} \ |\ \bm{\Theta}) = p(\bm{X}\ |\ \bm{Z}, \bm{\Theta}) p(\bm{Z}\ |\ \bm{\Theta})\\
			&l(\bm{\Theta}; \bm{X}, \bm{Z}) = \log \mathcal{L}(\bm{\Theta}; \bm{X}, \bm{Z}) =\\
			&= \log \prod_{\bm{q} \in \{0, 1\}^{N}} \left(\prod_{c=1}^{M} \theta_{c}^{\widehat{a}_c(\bm{q})}(1-\theta_{c})^{\widehat{b}_c(\bm{q})} \prod_{i=1}^{N} p_i^{q_i}(1-p_i)^{1-q_i}\right)^{\mathbb{I}\{\bm{Z} = \bm{q}\}} =\\
			&=\sum_{\bm{q} \in \{0, 1\}^{N}} \mathbb{I}\{\bm{Z} = \bm{q}\} \left( \sum_{c=1}^{M}\sum_{i=1}^{N} \widehat{a}_{c,i}(\bm{q}) \log \theta_{c} + \widehat{b}_{c,i}(\bm{q}) \log (1-\theta_{c})\right) +\\
			&+ \sum_{\bm{q} \in \{0, 1\}^{N}} \mathbb{I}\{\bm{Z} = \bm{q}\} \left(\sum_{i=1}^{N} q_{i} \log p_{i} + (1 - q_{i}) \log (1 - p_{i}) \right)
	\end{align*}
	Изменением порядка суммирования можно показать, что каждая из этих двух сумм распадается на $N$ сумм поменьше, по одной на каждую из скрытых переменных. В следствие этого и того, что компоненты случайного вектора $ \bm{Z} $ независимы в совокупности, шаги EM-алгоритма имеют вид:\\
	\noindent \textbf{E-шаг:}
	\begin{align*}
		&p(\bm{Z}\ |\ \bm{X}, \bm{\Theta}^{(t)})\ \propto\ p(\bm{X}\ |\ \bm{Z}, \bm{\Theta}^{(t)}) p(\bm{Z}\ |\ \bm{\Theta}^{(t)}) \implies\\
		&\implies \E_{\bm{Z}|\bm{X}, \bm{\Theta}^{(t)}} l(\bm{\Theta}; \bm{Z}, \bm{X}) = \sum_{i=1}^{N} \E_{\bm{z}_i|\bm{X}_i, \bm{\Theta}^{(t)}} \log \mathcal{L}(\bm{\Theta}; \bm{z}_i, \bm{X}_i) =\\
		&= \sum_{i=1}^{N} \sum_{q_i = 0}^{1} p(\bm{z}_i = q_i\ |\ \bm{X}_i, \bm{\Theta}^{(t)}) \left(
		\begin{aligned}
			&\sum_{c=1}^{M} \left[ \widehat{a}_{c,i}(q_i) \log \theta_{c} + \widehat{b}_{c,i}(q_i) \log (1 - \theta_{c}) \right] +\\
			&+ \log p(\bm{z}_i = q_i\ |\ \bm{\Theta})
		\end{aligned} \right) =\\
		&= \sum_{i=1}^{N} \left[ 
		\begin{aligned}
			&\gamma_{i,1}^{(t)} \left( \sum_{c=1}^{M} \left[a_{c,i} \log \theta_c + b_{c,i} \log(1-\theta_c) \right] + \log p_i \right) +\\ 
			&+ \gamma_{i,0}^{(t)} \left( \sum_{c=1}^{M} \left[b_{c,i} \log \theta_c + a_{c,i} \log(1-\theta_c) \right] + \log (1 - p_i) \right) 
		\end{aligned}
		\right] = Q(\bm{\Theta}\ |\ \bm{\Theta}^{(t)})
	\end{align*}
	
	\noindent \textbf{M-шаг:}
	\begin{gather*}
		p_{i}^{(t+1)} = \arg\max_{p_{i}} Q(\bm{\Theta}\ |\ \bm{\Theta}^{(t)}) \iff \frac{\gamma_{i, 1}^{(t)}}{p_{i}^{(t+1)}} - \frac{\gamma_{i, 0}^{(t)}}{1 - p_{i}^{(t+1)}} = 0 \iff p_{i}^{(t+1)} = \gamma_{i, 1}^{(t)}\\
		\begin{aligned}
			\theta_{c}^{(t+1)} = \arg\max_{\theta_c} Q(\bm{\Theta}\ |\ \bm{\Theta}^{(t)}) &\iff \frac{\sum_{i=1}^{N} \gamma_{i,1}^{(t)} a_{c, i} + \gamma_{i,0}^{(t)} b_{c, i}}{\theta_{c}^{(t+1)}} - \frac{\sum_{i=1}^{N} \gamma_{i,1}^{(t)} b_{c, i} + \gamma_{i,0}^{(t)} a_{c, i}}{1 - \theta_{c}^{(t+1)}} = 0\\
			&\iff \theta_c^{(t+1)} = \frac{\sum_{i=1}^{N} \gamma_{i,1}^{(t)} a_{c, i} + \gamma_{i,0}^{(t)} b_{c, i}}{\sum_{i=1}^{N} a_{c, i} +  b_{c, i}}
		\end{aligned}
	\end{gather*}
	Где необходимое условие локального экстремума является также достаточным в силу выпуклости функции $Q(\bm{\Theta}\ |\ \bm{\Theta}^{(t)})$ (\cite{MurphyProbabilisticML}, стр. 363-364).
\end{proof}
Стоит отметить, что на практике $p_{i}^{(t+1)}$ следует считать по эквивалетной, но уже численно устойчивой формуле:
\begin{gather*}
	p_{i}^{(t+1)} = \left(1 + \exp\left[\log(1 - p_{i}^{(t)}) - \log(p_{i}^{(t)}) + \sum_{c=1}^{M} \Delta_{c, i} (\log(\theta_{c}^{(t)}) - \log(1 - \theta_{c}^{(t)}))\right]\right)^{-1}
\end{gather*}
Где $\Delta_{c, i} := b_{c, i} - a_{c, i}$, а показатель экспоненты стоит искусственно приводить к диапазону $[-C; C]$ для некоторого $C > 0$ (авторами было выбрано $C = 100$). В противном случае $\prod_{c=1}^{M} (\theta_{c}^{(t)})^{a_{c, i}} (1 - \theta_{c}^{(t)})^{b_{c, i}}$ может представлять собой произведение тысяч или даже миллионов очень маленьких величин в больших степенях. Стандартной реализации чисел с плавающей запятой двойной точности недостаточно для хранения результатов промежуточных вычислений при использовании наивной формулы. 
\subsection{Первоначальная версия XClone: только ASE-модуль}
\subsubsection{Plate notation}
\subsubsection{Семплирование по Гиббсу}
\subsubsection{Предложенная модель, её недостатки}
\subsubsection{Identifiability problem, её решение в частном случае через сведение к задаче поиска соврешенного паросочетания}
\subsection{Заключительная версия XClone: ASE- и RDR-модули}
\subsubsection{Вариационный байесовский вывод}
\subsubsection{Структура ASE-модуля}
\subsubsection{Структура RDR-модуля}
\subsubsection{Сравнение модели с аналогами — CellRanger и CHISEL}
\subsubsection{Известные недостатки и планы по их исправлению}
